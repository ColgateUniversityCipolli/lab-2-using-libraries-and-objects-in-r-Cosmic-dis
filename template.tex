\documentclass{article}\usepackage[]{graphicx}\usepackage[]{xcolor}
% maxwidth is the original width if it is less than linewidth
% otherwise use linewidth (to make sure the graphics do not exceed the margin)
\makeatletter
\def\maxwidth{ %
  \ifdim\Gin@nat@width>\linewidth
    \linewidth
  \else
    \Gin@nat@width
  \fi
}
\makeatother

\definecolor{fgcolor}{rgb}{0.345, 0.345, 0.345}
\newcommand{\hlnum}[1]{\textcolor[rgb]{0.686,0.059,0.569}{#1}}%
\newcommand{\hlsng}[1]{\textcolor[rgb]{0.192,0.494,0.8}{#1}}%
\newcommand{\hlcom}[1]{\textcolor[rgb]{0.678,0.584,0.686}{\textit{#1}}}%
\newcommand{\hlopt}[1]{\textcolor[rgb]{0,0,0}{#1}}%
\newcommand{\hldef}[1]{\textcolor[rgb]{0.345,0.345,0.345}{#1}}%
\newcommand{\hlkwa}[1]{\textcolor[rgb]{0.161,0.373,0.58}{\textbf{#1}}}%
\newcommand{\hlkwb}[1]{\textcolor[rgb]{0.69,0.353,0.396}{#1}}%
\newcommand{\hlkwc}[1]{\textcolor[rgb]{0.333,0.667,0.333}{#1}}%
\newcommand{\hlkwd}[1]{\textcolor[rgb]{0.737,0.353,0.396}{\textbf{#1}}}%
\let\hlipl\hlkwb

\usepackage{framed}
\makeatletter
\newenvironment{kframe}{%
 \def\at@end@of@kframe{}%
 \ifinner\ifhmode%
  \def\at@end@of@kframe{\end{minipage}}%
  \begin{minipage}{\columnwidth}%
 \fi\fi%
 \def\FrameCommand##1{\hskip\@totalleftmargin \hskip-\fboxsep
 \colorbox{shadecolor}{##1}\hskip-\fboxsep
     % There is no \\@totalrightmargin, so:
     \hskip-\linewidth \hskip-\@totalleftmargin \hskip\columnwidth}%
 \MakeFramed {\advance\hsize-\width
   \@totalleftmargin\z@ \linewidth\hsize
   \@setminipage}}%
 {\par\unskip\endMakeFramed%
 \at@end@of@kframe}
\makeatother

\definecolor{shadecolor}{rgb}{.97, .97, .97}
\definecolor{messagecolor}{rgb}{0, 0, 0}
\definecolor{warningcolor}{rgb}{1, 0, 1}
\definecolor{errorcolor}{rgb}{1, 0, 0}
\newenvironment{knitrout}{}{} % an empty environment to be redefined in TeX

\usepackage{alltt}
\usepackage{amsmath} %This allows me to use the align functionality.
                     %If you find yourself trying to replicate
                     %something you found online, ensure you're
                     %loading the necessary packages!
\usepackage{amsfonts}%Math font
\usepackage{graphicx}%For including graphics
\usepackage{hyperref}%For Hyperlinks
\usepackage[shortlabels]{enumitem}% For enumerated lists with labels specified
                                  % We had to run tlmgr_install("enumitem") in R
\hypersetup{colorlinks = true,citecolor=black} %set citations to have black (not green) color
\usepackage{natbib}        %For the bibliography
\setlength{\bibsep}{0pt plus 0.3ex}
\bibliographystyle{apalike}%For the bibliography
\usepackage[margin=0.50in]{geometry}
\usepackage{float}
\usepackage{multicol}

%fix for figures
\usepackage{caption}
\newenvironment{Figure}
  {\par\medskip\noindent\minipage{\linewidth}}
  {\endminipage\par\medskip}
\IfFileExists{upquote.sty}{\usepackage{upquote}}{}
\begin{document}

\vspace{-1in}
\title{Lab XX -- MATH 240 -- Computational Statistics}

\author{
  First Author \\
  Affiliation  \\
  Department  \\
  {\tt email@domain}
}

\date{}

\maketitle

\begin{multicols}{2}
\begin{abstract}
This lab was split into two parts, one part taught me how to retrieve and process data by building a Batch file and the other part was about processing \texttt{JSON} files and working with their data. 
\end{abstract}

\noindent \textbf{Keywords:} loops; conditionals; indexing, data

\section{Introduction}

This lab was the first of a series of labs that aim to answer an important question. This question is, who of the three bands, The Front Bottoms, Manchester Orchestra, and All Get Out, contributed the most to a famous song called "Allen Town" in which they all collaborated on. The focus of this lab was to get comfortable processing and evaluating data.


\subsection{Intro Subsection}
You might need/want to discuss the topics in subsections. Or, you may have multiple questions.


\section{Methods}
The lab began by requiring us to build a Batch file for data processing. I was given fake music data to work with, but I needed functions that would allow me to manipulate the titles of the files and directories in a way that allowed me to better organize the data for evaluation later on. Specifically, I needed an R package that would allow me to manipulate strings (directory names, file names, etc.) easily. I used the \texttt{stringr} package to gain access to the functions \texttt{str\_sub()}, \texttt{str\_count()} and \texttt{str\_split()} which allowed me to work with the directory and file names to create the Batch file. \\

\noindent Later on, I had to  work with a \texttt{JSON} file, which required me to download a package, \texttt{jsonlite}, that would allow me to read the data in the \texttt{JSON} file. This package gave me access to the \texttt{fromJSON()} function which I used to load the \texttt{JSON} file into my \texttt{R} script and then read the data within the file. 


\subsection{Methods Subsection}
In collecting the data from the fake music file, I had to utilize for loops and conditional statements to specify what files I wanted to extract and what directories I wanted to look through. 

\section{Results}
Tie together the Introduction -- where you introduce the problem at hand -- and the methods --  what you propose to do to answer the question. Present your data, the results of your analyses, and how each reported aspect contributes to answering the question. This section should include table(s), statistic(s), and graphical displays. Make sure to put the results in a sensible order and that each result contributes a logical and developed solution. It should not just be a list. Avoid being repetitive. 


For the first part of the lab, I utilized the \texttt{stringr} functions to isolate the band name, album name, and track name from the files in the fake music folder. Each file in the fake music folder consisted of sub-folders named after fake artists, and the WAV files within these sub-folders were named using fake album and track names. I had to extract the individual names of the band, album, and track and paste them into a vector according to this template \texttt{streaming_extractor_music.exe "the track file" "the output filename with the combined artist, album, and track names}

This helped immensely during the \texttt{JSON} section of the lab where it required me to loop through all the contents of my lab 2 folder and specify that I only wanted to work through the \texttt{JSON} file. 

\subsection{Results Subsection}
Subsections can be helpful for the Results section, too. This can be particularly helpful if you have different questions to answer. 


\section{Discussion}
 You should objectively evaluate the evidence you found in the data. Do not embellish or wish-terpet (my made-up phase for making an interpretation you, or the researcher, wants to be true without the data \emph{actually} supporting it). Connect your findings to the existing information you provided in the Introduction.

Finally, provide some concluding remarks that tie together the entire paper. Think of the last part of the results as abstract-like. Tell the reader what they just consumed -- what's the takeaway message?

%%%%%%%%%%%%%%%%%%%%%%%%%%%%%%%%%%%%%%%%%%%%%%%%%%%%%%%%%%%%%%%%%%%%%%%%%%%%%%%%
% Bibliography
%%%%%%%%%%%%%%%%%%%%%%%%%%%%%%%%%%%%%%%%%%%%%%%%%%%%%%%%%%%%%%%%%%%%%%%%%%%%%%%%
\vspace{2em}

\noindent\textbf{Bibliography:} Note that when you add citations to your bib.bib file \emph{and}
you cite them in your document, the bibliography section will automatically populate here.

\begin{tiny}
\bibliography{bib}
\end{tiny}
\end{multicols}

%%%%%%%%%%%%%%%%%%%%%%%%%%%%%%%%%%%%%%%%%%%%%%%%%%%%%%%%%%%%%%%%%%%%%%%%%%%%%%%%
% Appendix
%%%%%%%%%%%%%%%%%%%%%%%%%%%%%%%%%%%%%%%%%%%%%%%%%%%%%%%%%%%%%%%%%%%%%%%%%%%%%%%%
\newpage
\onecolumn
\section{Appendix}

If you have anything extra, you can add it here in the appendix. This can include images or tables that don't work well in the two-page setup, code snippets you might want to share, etc.

\end{document}
